% Please do not change the document class
\documentclass{scrartcl}

% Please do not change these packages
\usepackage[hidelinks]{hyperref}
\usepackage[none]{hyphenat}
\usepackage{setspace}
\doublespace

% You may add additional packages here
\usepackage{amsmath}

% Please include a clear, concise, and descriptive title
\title{Evaluation}

% Please do not change the subtitle
\subtitle{COMP230 - Evaluation}

% Please put your student number in the author field
\author{1506530}


\begin{document}
	
\maketitle

\section{Introduction}

During semester two, there have been some good points, for example time keeping has been efficient. Though the projects may not have made the progress that was hoped for and it may seem like time management was an issue it has been better than the previous year. Along with this communication and organization has been better, mainly with the comp230 project. Many issues have arisen five of those are: motivation, self confidence, asking for help, patience and perfectionism.

\section{Motivation}

This semester has been difficult to keep the motivation going, theres no surety whether it was the difficulty of the graphics module or the lack of interest in it. Its difficult to say which though the solution will be the same. Over the next semester priorities will be taken with a greater weight, so if there is a module that will result in a greater benefit than another, more effort will be put into that one. This does have its drawbacks, for example it could lead to not attempting another module. If this method does not work then an attempt will be made to engage with others to improve the 'Fun' aspect of the course.

\section{Self confidence}

Self confidence in the work created over this semester was poor, not so much for the comp230 though still present it could not compare to the lack of confidence present in the other two modules. The constant self criticism of the work as well as the feeling of not being good enough to be a programmer. Thinking back to the first year a point to remember is that though the work may get difficult and the road rough, one must stay determined and they will reach their goal. The first step that will be taken is to keep trying, along with this engaging with peers and keeping the tutors up to date with the modules progress will help keep in mind that everyone has their struggles and self confidence in their work tends to be a large issue. If one fails, one must get up and try again. 

\section{Asking for Help}

Over the last semester asking for help has been a major issue, only one pull request was made and even then it was for the peer review which went less than satisfactory. This was an issue last year and is getting tiresome, the issue is that when everyone seems to be excelling in their respective modules its difficult to ask for help as it feels as though one accepts defeat. Though it feels this way, a solution must be found to stop this as it is a major contributor to the lack of quality in the work. One solution that has been thought of is to ensure a pull request along with one question to be submitted each week, even if work has not been made. This will help to make at least some progress and clear up any issue every week.

\section{Patience}

Over the last year, patience with work has been lacking considerably. When working, a lot of time was spent trying to calm down after loosing control while working on a project. This was believed to be cause by a feeling of being lost, this coming from a lack of progress and possibly understanding around the module. Trying to overcome this has been difficult as each attempt resulted in loosing patients with the current project, this has lead to further frustration and a complete lack of progress. That was until the pressure of deadlines came along and created a certain level of focus that allowed the completion of the modules to a passing level. Theres no clear solution to this issue, as its not something that can be solved over a short period, a plan to improve this issue must be undertaken over a period of years and worked on throughout ones life. Though to take a step towards solving this would be to take more breaks more often when working on a task and not be so bull headed by tring to finish it all in one session.

\section{Perfectionism}

Perfectionism is something that keeps crawling up, while trying to complete a module an attempt is made at making it perfect or it is not considered good enough. This trait tends to be conflicted greatly with the other issues that have been encountered, trying to create the 'perfect' project while at the same time trying to find the motivation, self confidence and not to loose patients is proving to be less than easy. It has created a vicious circle that needs to be broken, once the other issues are addressed it is possible that this one could be solved as well. This solution is not perfect, but as it stands another solution is not clear at this time. Perhaps if a greater priority was given about what needed to be done and a lesser concentration on what is hoped to be done, a higher quality of work can be achieved by ensuring that tasks that matter the most are done to a higher quality.

\section{Final Remarks}

The semester has not been as productive and 'fun' as was hoped, although a lot has been learnt about how to deal with certain issues. It is hoped that with more work on these personal traits that they can be solved permanently. It is hoped that next semester will provide a greater depth of learning for programming and personal development than this semester, in which a great feeling of being lost was present throughout.

\bibliographystyle{ieeetran}
\bibliography{comp230_4_1506530}

\end{document}